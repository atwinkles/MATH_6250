\documentclass{article}
\usepackage[utf8]{inputenc}
\usepackage{amsmath}
\usepackage{amsthm}
\usepackage{amsfonts}
\usepackage{amssymb}
\usepackage{graphicx}
\usepackage{bbold}
\usepackage{url}
\usepackage{booktabs}
\usepackage{geometry}
\usepackage{marvosym}
\usepackage{wasysym}
\usepackage{fancyhdr}
\usepackage{gensymb}

\usepackage[scaled]{beramono}
\usepackage{listings}
\lstset{
    language=Python,
    showstringspaces=false,
    formfeed=\newpage,
    tabsize=4,
}
\newcommand{\code}[2]{
    \hrulefill
    \subsection*{#1}
    \lstinputlisting{#2}
    \vspace{2em}
}

\pagestyle{fancy}
\fancyhf{}
\rhead{MATH 6250}
\lhead{Alexander Winkles}
\chead{\Large \textbf{Homework 1}}
\cfoot{Page \thepage}

\title{Homework 1}
\author{Alexander Winkles}
\date{1 February 2016}

\begin{document}

The goal of this homework assignment was to create several SO(3) rotation matrices for a cube, then create a program that will randomly multiply said matrices until all the elements of their group are found.
All computations were done in Python using Numpy. 
The output of the code gave the following results:
\begin{enumerate}
\item The identity matrix:
\[
\begin{bmatrix}
1 & 0 & 0\\
0 & 1 & 0\\
0 & 0 & 1
\end{bmatrix}
\]

\item Rotations about axes (90\degree):\\
\textit{Z axis:}\\
\begin{equation*}
90  = 
\begin{bmatrix}
0 & -1 & 0\\
1 & 0 & 0\\
0 & 0 & 1
\end{bmatrix}
\quad
180 = \begin{bmatrix}
-1 & 0 & 0\\
0 & -1 & 0\\
0 & 0 & 1
\end{bmatrix}
\quad
270 = \begin{bmatrix}
0 & 1 & 0\\
-1 & 0 & 0\\
0 & 0 & 1
\end{bmatrix}
\end{equation*}

\textit{Y axis:}\\
\begin{equation*}
90 = \begin{bmatrix}
0 & 0 & -1\\
0 & 1 & 0\\
1 & 0 & 0
\end{bmatrix}
\quad
180 = \begin{bmatrix}
-1 & 0 & 0\\
0 & 1 & 0\\
0 & 0 & -1
\end{bmatrix}
\quad
270 = \begin{bmatrix}
0 & 0 & 1\\
0 & 1 & 0\\
-1 & 0 & 0
\end{bmatrix}
\end{equation*}

\textit{X axis:}\\
\begin{equation*}
90 = \begin{bmatrix}
1 & 0 & 0\\
0 & 0 & 1\\
0 & -1 & 0
\end{bmatrix}
\quad
180 = \begin{bmatrix}
1 & 0 & 0\\
0 & -1 & 0\\
0 & 0 & 1
\end{bmatrix}
\quad
270 = \begin{bmatrix}
1 & 0 & 0\\
0 & 0 & -1\\
0 & 1 & 0
\end{bmatrix}
\end{equation*}

\item Rotations about corners (120\degree):\\
Four corners can be rotated about: 1, 2, 3, 4.
\begin{enumerate}
\item Corner 1\\
\begin{equation*}
120 = \begin{bmatrix}
0 & -1 & 0\\
0 & 0 & -1\\
1 & 0 & 0
\end{bmatrix}
\quad
240 = \begin{bmatrix}
0 & 0 & 1\\
-1 & 0 & 0\\
0 & -1 & 0
\end{bmatrix}
\end{equation*}

\item Corner 2\\
\begin{equation*}
120 = \begin{bmatrix}
0 & 1 & 0\\
0 & 0 & 1\\
1 & 0 & 0
\end{bmatrix}
\quad
240 = \begin{bmatrix}
0 & 0 & 1\\
1 & 0 & 0\\
0 & 1 & 0
\end{bmatrix}
\end{equation*}

\item Corner 3\\
\begin{equation*}
120 = \begin{bmatrix}
0 & -1 & 0\\
0 & 0 & 1\\
-1 & 0 & 0
\end{bmatrix}
\quad
240 = \begin{bmatrix}
0 & 0 & -1\\
-1 & 0 & 0\\
0 & 1 & 0
\end{bmatrix}
\end{equation*}

\item Corner 4\\
\begin{equation*}
120 = \begin{bmatrix}
0 & 0 & -1\\
1 & 0 & 0\\
0 & -1 & 0
\end{bmatrix}
\quad
240 = \begin{bmatrix}
0 & 1 & 0\\
0 & 0 & -1\\
-1 & 0 & 0
\end{bmatrix}
\end{equation*}

\end{enumerate}

\item Edge rotations (180\degree):\\
\begin{equation*}
1,2 = \begin{bmatrix}
0 & 0 & 1\\
0 & -1 & 0\\
1 & 0 & 0
\end{bmatrix}
\quad
1,3 = \begin{bmatrix}
0 & -1 & 0\\
-1 & 0 & 0\\
0 & 0 & -1
\end{bmatrix}
\quad
2,4 = \begin{bmatrix}
0 & 1 & 0\\
1 & 0 & 0\\
0 & 0 & -1
\end{bmatrix}
\end{equation*}

\begin{equation*}
3,4 = \begin{bmatrix}
0 & 0 & -1\\
0 & -1 & 0\\
-1 & 0 & 0
\end{bmatrix}
\quad
1,5 = \begin{bmatrix}
-1 & 0 & 0\\
0 & 0 & -1\\
1 & 0 & 0
\end{bmatrix}
\quad
2,6 = \begin{bmatrix}
-1 & 0 & 0\\
0 & 0 & 1\\
0 & 1 & 0
\end{bmatrix}
\end{equation*}

\end{enumerate}

The program, after computing all of these matrices, then computed the determinante of each one to check if they are all in SO(3).
After this, it counted all of the matrices and printed them out.
In addition, the code help a matrix representing the cube, which was used for determining which matrices corresponded to which rotations.
\textbf{The group of rotations consists of 24 unique matrices, all of which are SO(3).}

For the bonus, the following new rotation was introduced to the group that was not computed initially in order to create a new, larger group:
\[
\begin{bmatrix}
\cos(175\degree) & -\sin(175\degree) & 0\\
\sin(175\degree) & \cos(175\degree) & 0\\
0 & 0 & 1
\end{bmatrix}
\]
After doing several trials, one with 10,000 iterations, one with 20,000 iterations, and one with a 1,000,000 iterations, it was concluded that a new group was not formed,
or if one was, it was an infinite group.
The first trial created 9917 matrices, while the second created 19960 matrices.
The third trial was too taxing for the student's computer, so was prematurely ended. 
Another trial was conducted with a new angle of 170\degree, which yielded similar results.

\clearpage

\section*{Appendix}

\code{Cube Rotations Code}{rotation.py}

\end{document}
