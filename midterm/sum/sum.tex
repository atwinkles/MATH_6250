\documentclass{article}
\usepackage[utf8]{inputenc}
\usepackage{amsmath}
\usepackage{amsthm}
\usepackage{amsfonts}
\usepackage{amssymb}
\usepackage{amstext}
\usepackage{gensymb}
\usepackage{graphicx}
%\usepackage{bbold}
%\usepackage{url}
%\usepackage{booktabs}
%\usepackage{marvosym}
%\usepackage{wasysym}
\pagenumbering{arabic}
\usepackage{fancyhdr}
\usepackage[margin=1.0in]{geometry}
\usepackage{eucal}
\usepackage{multicol}
\usepackage{verbatim}

\pagestyle{fancy}
\fancyhf{}
\rhead{MATH 6250}
\lhead{Alexander Winkles}
\chead{\Large \textbf{Midterm Summary}}
\cfoot{Page \thepage}

\begin{document}

Weiner's goal in his paper ``A Spherical Fabricius-Bjerre Formula With Applications to Closed Space Curves" 
is to take Fabricius-Bjerre's classical result 
\begin{equation}
t-s = d + i
\end{equation}
or, for a closed curve C the difference between the number of exterior and interior double tangents is equal to 
the number of double points and half the number of inflection points 
and translate it for closed spherical curves.
He does so by defining what these items would be in terms of spherical curves (curves on the surface of a sphere)
and defining a new variable counting the antipodal points of a curve $\gamma$.
By doing so, he proves the new theorem that
\begin{equation}
t - s = d - a + i,
\end{equation}
where a is the number of antipodal pairs of $\gamma$. 
From here, he applies this theorem to space curves called $\alpha$, which leads to the formula
\begin{equation}
i = t - s - d + a
\end{equation}
where
\begin{align*}
2i &= \textrm{the number of vertices of $\alpha$}\\
d &= \textrm{the number of pairs of direcetly parallel tangents of $\alpha$}\\
a &= \textrm{the number of pairs of oppositely parallel tangents of $\alpha$}\\
t &= \textrm{the number of pairs of concordant parallel osculating planes $\alpha$}\\
s &= \textrm{the number of pairs of discordant parallel osculating planes of $\alpha$}
\end{align*}
Weiner concludes his paper with two corollaries dealing with generic non-degenerate closed space curvees with torsion.
\newline

\underline{Bonus Paragraph}\newline

The main theorem of Weiner's paper is that which is described by Equation (2) in the above paragraph:
\newtheorem{weiner}{Theorem}
\begin{weiner}
Let $\gamma: C \to S$ be a generic closed spherical curve; then
\begin{equation*}
t - s = d - a + i
\end{equation*}
\end{weiner}
To do this, he defines terms similar to those in the original Fabricius-Bjerre theorem, with the addition of a term for the antipodal pairs of $\gamma$.
He replaces the signed half-tangents with half-geodesics and considers the change in the number of points the signed geodesics have in comment with the spherical curve, denoted as $N(x)$.
Likewise, he observes that $N(x)$ changes as $\gamma$ passes either point of an antipodal pair of $\gamma$. 
This consideration adds the -a term in the new formula.


\end{document}
