\documentclass{article}
\usepackage[utf8]{inputenc}
\usepackage{amsmath}
\usepackage{amsthm}
\usepackage{amsfonts}
\usepackage{amssymb}
\usepackage{amstext}
\usepackage{gensymb}
\usepackage{graphicx}
%\usepackage{bbold}
%\usepackage{url}
%\usepackage{booktabs}
%\usepackage{marvosym}
%\usepackage{wasysym}
\pagenumbering{arabic}
\usepackage{fancyhdr}
\usepackage[margin=1.0in]{geometry}
\usepackage{eucal}
\usepackage{multicol}
\usepackage{verbatim}

\pagestyle{fancy}
\fancyhf{}
\rhead{MATH 6250}
\lhead{Alexander Winkles}
\chead{\Large \textbf{Midterm}}
\cfoot{Page \thepage}

\begin{document}

In Weiner's paper ``A Spherical Fabricius-Bjerre Formula with Applications to Closed Space Curves,'' he translates Fabricius-Bjerre's clasically result
\begin{equation*}
t - s = d +i
\end{equation*}
for closed plane curves to describe curves embedded on a unit sphere. 
Originally, Fabricius-Bjerre derived his formula by defining interior and exterior double tangents (t and s), double points (d), and half inflection points (i)
and observed the change in how many points of the curve C intersect the tangent of the curve at a given point as the point moved along the curve and more specifically
through double points, inflection points, and double tangents.
Weiner adapts this proof to spherical curves with the additional consideration of antipodal points, or points on the opposite sides of a sphere.
Past this, the author applies his formula to closed curves in 3-space and develops ideals dealing with generic (to be defined) non-degenerate closed curves with torsion.

To begin with, Weiner defines $\gamma: C \to S$ to be a $\mathcal{C}^3$ immersion of the circle C onto the unit sphere S, which is called a closed spherical curve.
Let $\gamma'$ be the field of unit tangent vectors pointing in the direction $\gamma$ travels.
In order to translate Fabricius-Bjerre's results to closed spherical curves, Weiner defined double points, antipodal pairs, inflection points, and double tangents for $\gamma$.
In his paper, a point $P \in S$ is a double point of $\gamma$ if $\gamma^{-1}$(P) contains multiple inputs from the circle C. 
This is visualized as a closed loop within the curve and is restricted to being generated from only two points from C for the sake of simplicity.
More so, it is restricted so $\gamma'(x) \neq \pm \gamma'(y)$, where $x,y \in C$ are the points which map to the double point. 
This restriction is imposed so that it is a genuine double point with different tangents, rather than a line that crosses itself (and would thus have an equal tangent up to a minus sign).
Next, for $P \in S$, $\bar{P}$ is defined as P's antipode.
Thus, ${P,\bar{P}}$ is an antipodal pair of points for $\gamma$ if $\exists x,y \in C$ such that $\gamma(x) = P$ and $\gamma(y) = \bar{P}$.
It is assumed that each pair ${P,\bar{P}}$ is not a double point so no ``unique'' points are counted twice.
Likewise, as before $\gamma'(x) \neq \pm \gamma'(y)$, to restrict them from being the same line. %WHAT HELP PLS


\end{document}
